\documentclass{article}

\usepackage{fullpage}
\usepackage{amsmath}
\usepackage{amssymb}
\usepackage{amsthm}
\usepackage{multirow}
\usepackage{graphicx}
\usepackage{mathtools}
\usepackage{pifont}
\usepackage{color}
%\usepackage{biblatex}
\usepackage{setspace}
\usepackage{blindtext}
%\onehalfspacing


\newtheorem{theorem}{Theorem}
\newtheorem{proposition}{Proposition}
\newtheorem{corollary}{Corollary}
\newtheorem{conjecture}{Conjecture}
\newtheorem{lemma}{Lemma}
\theoremstyle{definition}
\newtheorem{definition}{Definition}

\DeclareMathOperator*{\argmin}{argmin}

%%%% New commands %%%%
\def\Pr{\mbox{Pr}}
\def\PR{\mbox{Pr}}
\newcommand{\yes}{\operatorname{yes}}
\newcommand{\no}{\operatorname{no}}
\newcommand{\tilf}{\tilde{f}}
\newcommand{\bias}{\tilf}
\newcommand{\B}{\mbox{Big}}
\newcommand{\vect}[1]{\boldsymbol{#1}}

\newcommand{\tick}[0]{\checkmark}

\newcommand{\nicolasc}{\ding{110}\ding{43}\textcolor{blue}}
\newcommand{\benjamin}{\ding{110}\ding{43}\textcolor{green}}

\pagenumbering{gobble}

\title{(Im)balances of Power in the Age of Personal Data}
\author{Principal Investigator: Paul Laskowski\\
paul@ischool.berkeley.edu}
\begin{document}

\maketitle

\subsection*{Grant Category: Multi-year}
I would appreciate consideration of a one-year grant if the multi-year grant is not awarded.

\subsection*{Additional Key Personnel}
Benjamin Johnson\\benjaminejohnson@gmail.com

\subsection*{Project Proposal Abstract}
Using technologies that record and analyze our purchases and behaviors, businesses are presenting us with tremendous opportunities, but also eroding our expectation of privacy.  To assess the true cost of personal data collection, we must look beyond individual decisions to how the cumulative impact of many citizens influences technology and institutions.   Improvements in data collection may carry risks that range from emboldened repressive elements in government, to a reduction in privacy choices by firms that seek to profile our preferences.

This proposal examines the role of citizens as a whole, especially in relation to governments and corporations. We seek to measure how individual privacy choices tilt the playing field that all citizens face. The ultimate goal is a more complete and more accurate assessment of the costs of personal data collection, one that may benefit the public dialogue about data collection and inform future policy discussions.

Our primary research methods will involve the development of game-theoretic economic models describing interactions between consumers, governments, corporations and other potential actors.  These will be complemented by human-subject experiments that seek to understand why some uses of personal data are adopted, and why some are widely opposed.  After examining current trends in personal data collection, we will attempt to extrapolate them far beyond the present day.  The ultimate deliverable for the project will be a report outlining possibilities and accompanying consequences for personal data exchange markets 30 years into the future.


\subsection*{Key Deliverables}
\begin{itemize}
\item A characterization of the effects that personal data collection can have on the balance of power between citizens, governments, and corporations, as identified through game theoretic modeling.
\item An assessment of potential factors that affect the evolution of practices related to personal data, as measured by human-subject experiments.
\item Six or more conference papers and two or more journal articles that disseminate our findings to the academic community.
\item One extensive report on 30-year future scenarios of personal data markets.
\end{itemize}
\newpage

\pagenumbering{arabic}

\subsection*{Preface to the Revised Proposal}

The scoping award provided by CLTC has given my research team the opportunity to further our analyses and refine the goals of our research initiative.  I would like to highlight three main achievements from this time that are reflected in this new proposal.

\begin{enumerate}
\item \textbf{Modeling of Privacy and Legal Enforcement.}  
The use of personal data by governments is of special concern to marginalized groups of people.  Our latest modeling effort focuses on the way that modern data technologies can facilitate the disproportionate enforcement of laws against minority groups.  A major contribution of this study is a game theoretic framework that can distinguish between four different types of privacy: two that are technologically enforced, and two that are legally enforced.  We have already submitted one paper for publication based on this effort~\cite{johnsonprivacy}; and we hope this analysis will form a useful foundation for future investigations.  %This work culminated in a paper that is currently under submission~\cite{johnsonprivacy}.

\item \textbf{Dynamic Research Focus.}  The ultimate goal of our initiative is an understanding of the ways in which personal data technologies may be used (and abused) far into the future.  Creating this outlook demands that we examine the process by which personal data practices evolve over time.  Advances in technology often present a tension that is apparent in our models: Governments and corporations may perceive direct benefit in uses of personal data that conflict with individual notions of privacy.  Recognizing the importance of these issues, we added a new research question that is explicitly tied to the dynamic evolution of privacy.  This question will serve as a bridge to our final goal of understanding the role of personal data, 30 years into the future.


\item \textbf{Plans for Experimentation.} To support our dynamic research focus, we will investigate the evolution of privacy attitudes and data practices with a series of human-subject experiments.  Our first design explores the factors that cause individuals to oppose a use of personal data.  Participants will be presented with information regarding the use of an emerging technology, then asked to sign a petition that regulates the practice.  This framework allows us to explore the extent to which abuses of power lead to political opposition.  Our second design complements the first by asking how decision makers choose which uses of personal data to adopt.  We will ask subjects to imagine themselves in the role of government officials that must decide whether or not to deploy a new surveillance technology, depending on the nature of the threat to the government.  Taken together, we expect these experiments to complement our game theoretic analysis, broaden our view beyond purely rational responses, and yield a more complete picture of the way personal data practices may evolve over time.


\end{enumerate}

\subsection*{Introduction}

What truly matters when protecting our personal data?   We live in a world where our purchases, browsing history, and even mouse clicks are logged by computer systems and increasingly analyzed.  While businesses and governments have identified tremendous opportunities in data like this, how can we measure the corresponding costs associated with the privacy we are losing?  

One avenue is to consider individual decisions.  Economic models let us predict how much firms can price discriminate, based on their knowledge of a consumer~\cite{calzolari2006optimality, taylor2004consumer}.  Surveys give us insight into how much consumers value different types of personal information.  Meanwhile, behavioral studies reveal how consumers may give up their data for simple convenience, despite how much they say they value keeping it secret \cite{acquisti2007can,Grossklags:WEIS07,acquisti2004privacy}.

Yet a focus on individual decisions misses a large part of what makes privacy valuable, if not the most important part.  Excerpts from popular media reveal a deep concern over how mass collection of personal data will affect citizens as a whole, and the potential transfer of power to governments and corporations.  The Electronic Frontier Foundation's main page on privacy argues that ``national governments must put legal checks in place to prevent abuse of state powers."  Similarly, the Bloomberg \emph{Wired for Repression} series argues, ``In authoritarian countries throughout the Middle East and North Africa, Western surveillance tools have empowered repression.''  As Valentino-Devries, Singer-Vine and Soltani recently argued in the Wall Street Journal, increasingly targeted pricing by companies ``diminishes the Internet's role as an equalizer.'' 

Though sweeping in scope, we must grapple with risks like these if we are to assess the true cost of collecting personal data.  It is not enough to look at individual decisions in isolation because the cumulative impact of individual decisions can change institutions, laws, and practices.  Profiling individual consumers may become easier as more data is available.  Enhanced data collection may embolden repressive elements in government, increasing the risk for dissidents.  Consumers may face reduced privacy choices if companies can neglect privacy-conscious individuals.

This proposal therefore examines the role of citizens as a whole, especially in relation to governments, corporations, and other actors.  My team and I seek to measure how individual privacy choices may tilt the playing field that citizens face.   The ultimate goal is a more complete and more accurate assessment of the costs of personal data collection, one that may benefit the public dialogue about data collection and inform future policy discussions.

\subsection*{Research Question 1: How does the collection of personal data affect the balance of power between citizens and the state?}

Modern tools for data collection give considerable advantage to governments in monitoring the behavior of their own citizens~\cite{acquisti2009predicting}, and citizens are beginning to pay attention.  Once regarded chiefly as a problem of third-world autocracies, many western governments, including the United States, now face widespread scrutiny over the potential abuse of surveillance technology, by both national security and law enforcement agencies~\cite{schneier2013oppression}. � Document leaks and news reports have revealed an extensive surveillance apparatus, fueling a lively debate about the proper limits of governmental data collection~\cite{landau2013making,landau2014}.

%Proponents maintain that surveillance programs are instrumental in preventing terrorist attacks. Opponents counter that spying erodes individual privacy and facilitates totalitarian states.

In a 2014 study, I worked with colleagues at UC Berkeley to examine this issue through the lens of government incentives~\cite{laskowskigovernment}.  While abusive governments will certainly use surveillance technology to their advantage, a more subtle question is whether technology leads governments to be more or less repressive in the first place.  Using a stylized game theoretic model, we examined the incentive structure of a government that wishes to remain in power.  Our model showed that surveillance technology incentivizes greater abuse of power, since it increases the benefit associated with each instance of abuse.

In a follow-up study, which was made possible by our scoping award, we turned our attention to marginalized groups within society~\cite{johnsonprivacy}.  Such groups may be particularly sensitive to the collection of data by law enforcement agencies, given the possibility that they are disproportionately targeted for investigation.
%Such groups may be especially concerned by the collection of personal data by law enforcement agencies, because the technologies available to analyze personal data may  %.  Personal data technologies may 
%facilitate the identification of individuals based on group characteristics, and result in disproportionate targeting relative to the evidence that a crime has occured. %as well as the gathering of evidence that a crime has occurred.  
In our study, we defined a metric for measuring the divisiveness of a law to see whether privacy protection could lead to more equitable outcomes.  Our model distinguished between four different types of privacy protection: two that are legally enforced and two that are technologically enforced.  Each type of protection limited the impact of certain laws, though none was perfectly aligned with the goal of reducing the disproportionate targeting of minorities.


%To support our results, my team was careful to separate the use of surveillance technology from its abuse.  This reflects our understanding that the collecting of citizen data by governments is not inherently corrupt.  Moreover, many applications of surveillance, including for law enforcement and security, are uncontroversial.  We therefore adopted a working definition of abuse as the use of power \textit{in excess of moral or ethical standards}.  

%Our game theoretic model suggests that surveillance technology incentivizes greater abuse of power by governments.  Under surprisingly general conditions, stronger surveillance technology will lead a rational government that wishes to stay in power to abuse its power more.   Although we recognize that well-meaning government officials may ignore such incentives in order to behave ethically, we believe that incentives matter for explaining differences in abuse levels. 

While these initial results are promising, they open myriad directions for further study.  How does the surveillance strategy of a rational government depend on the nature of the threat to its power, be it a radicalized insurgency or a political opposition?
%How does a rational government behave in the presence of multiple threats to its power, say a radicalized insurgency and a political opposition?  
How can we best balance the interests of marginalized communities with the security of an entire society? What types of counter-technologies can tilt the playing field in favor of more benign governance?  By answering these questions, this initiative will shed light on how personal data technologies affect the relationship between a government and its citizens, and what technologies are most effective in promoting ethical governance. 

\subsection*{Research Question 2: How does the collection of personal data affect the balance of power between consumers and corporations?}

A second line of inquiry concerns the balance of power between consumers and corporations.  While a large body of works examines when consumers will reveal personal information to merchants, we understand far less about the processes that determine what options are available to consumers, and how they may change with advancements in tracking technology.

One area of rapid change is in the sharing of consumer data between firms.  Recognizing the economic advantages that can come from knowing an individual's past purchase decisions, %linking consumer decisions across different merchants, 
a new generation of firms has developed technologies to facilitate the exchange of such data between merchants. %rapidly lowering the barriers to %free has created a market for the 
%exchanging customer data between firms.

%monetizing personal data. %monetizing the sharing of consumer data. 
In an early paper, my coauthors and I applied game-theoretic modeling to a scenario in which consumers purchase goods from two different firms in sequence~\cite{johnsoncaviar}.  When firms are allowed to sell their purchase data to each other, they can leverage the information they learn about consumers to set discriminatory prices~\cite{odlyzko2003privacy}.
Like previous studies in this lineage~\cite{calzolari2006optimality, acquisti2005conditioning}, we find that the ultimate consequences for consumers depend on whether they are myopic, regarding each purchase in isolation, or fully rational, choosing products strategically in light of how purchases may affect future price offers.  While myopic consumers always lose out when firms share information, sophisticated consumers will %mitigate this effect by limiting 
limit the information they reveal through purchases.

Results like ours are suggestive of the changing landscape facing consumers, but they fall short of describing a world in which data collection is pervasive.  Once consumers reveal information about themselves, it may remain valuable to firms for years into the future.  Consumers may be risk averse, unsure of what goods they will want to buy into the future.  They may also find that protecting their privacy over time requires too much continued effort, and it is more practical to accept discriminatory practices.

To better understand these possibilities, we will prioritize the development of models with more than two firms and a continuum of consumers.  In the simplest possible setup, consumers and firms interact in an infinite series of time steps.  In each interval, consumers face a purchasing decision, and firms may be allowed to share the information they collect with each other.  Both consumers and firms value present payoffs more than future ones, but they may differ in their discount rates and risk preferences.  

Such a setup could investigate economic effects that have so far been absent from the literature.  While most results to date have been essentially neoclassical -- finding that more information increases welfare -- a time-based model would let us ask whether consumers face increasing pressure to avoid information-revealing purchases over time, with potential welfare-reducing consequences.  We will also investigate the distribution of welfare between firms and consumers -- do repeated purchases continue to erode consumer surplus toward zero, or can strategic consumers mitigate this effect?


\subsection*{Research Question 3: What factors best explain the evolution of attitudes and practices related to the exchange of personal data?}

A common theme in our research is the tension between the direct benefits a government or firm receives from a use of personal data and opposition from individuals.  Game theoretic modeling gives us a way to understand how these forces interact in an environment with strategic actors, but such models are uncalibrated in the sense that results are directional and we have no way of estimating the response to a particular new technology.  To reason about what personal data practices may be prevalent in the future, we need a more quantitative understanding of the factors that motivate citizens to oppose a practice, as well as those that motivate an institution to adopt it.  We will pursue this understanding with a series of randomized, human-subject experiments.  This methodology is well-suited to our problem space, since it allows us to measure causal effects without the endogeneity issues inherent in a study of personal data practices.

Our first experiment focuses on factors that cause citizens to oppose a use of personal data.   There is a large body of work that applies experiments to study attitudes towards privacy~\cite{acquisti2007can,Grossklags:WEIS07,acquisti2004privacy, horne2015privacy, xie2006volunteering, baker2012does, kugler2015surveillance}.  To study opposition in particular, we will build our design around the act of signing a petition.  Subjects will be presented with information about a new application of personal data, then invited to sign a petition that opposes it.  Guided by our game theoretic models, we will incorporate abuse of power as a key explanatory variable.  Subjects in one condition will only be given information about the invasiveness of a practice.  In a second condition, subjects will learn how abuses of power have negatively impacted individuals.  A control group will read about a practice that is not overtly invasive.

A petition framework has the advantage that subjects may perceive real-world consequences to their action.  We expect our measure to reflect not just a willingness to sign a petition, but political motivation in general.  Accordingly, we believe our results will generalize to democratic governments, firms, and even, to a lesser extent, authoritarian states that are sensitive to public opinion.



Our second experiment will address the perspective of a government that must decide what personal data to collect and how to use it.  Participants will be asked to imagine themselves in the role of government officials that are considering whether or not to deploy a newly developed surveillance technology that provides improved access to the behavior of all citizens.  Guided again by our game theoretic models, we plan to investigate how different threats to government power lead to different decisions.  Specifically, we will vary both the size of the opposition to the government, as well as its intensity, which may range from peaceful demonstration to armed insurrection.  By searching in this grid, we hope to shed light on the  dynamic interaction between abuse and opposition.


In designing our initial round of experiments, we plan to focus on a single technology: algorithmic facial recognition.  Central to our choice is the timing of this emerging technology.  Although several commercial products now feature facial recognition, most consumers have yet to grapple with the emerging consequences to their personal privacy.  Moreover, the legal system and cultural practices have yet to account for this technology's potential ubiquity.  The current timing gives us a rare opportunity to present information regarding facial recognition to citizens that have not yet formed a strong opinion.  The details that we choose to include may therefore have a greater impact on our subjects' attitudes.  Moreover, we may expect stories of the use (and abuse) of facial recognition to emerge in the next few years, allowing us to evaluate our results in light of actual evolving public attitudes.


%Our first two research questions are static in nature.  Through game theoretic analysis, we are able to see how the capabilities of a technology can affect balances of power.  Although data technologies advance over time, their capabilities are mediated by an evolving landscape of social factors.  These include the legal system, the use of counter-technologies, and cultural practices, all of which change in response to attitudes about privacy.  Therefore, to understand dynamic trends in the use of personal data, a central concern must be the way that these attitudes change over time.


\subsection*{Research Question 4: What economic structures for exchanging personal data could exist 30 years into the future?}

By combining game theoretic modeling with experimentation, my team and I will attempt to paint a more complete picture of current trends in personal data collection.  Our final research question attempts to extrapolate these trends well into the future.  In thirty years time, what structures could govern the flow of personally identifiable data?  Will consumers maintain any privacy in the face of advancing data collection technologies?  What relationships could exist between citizens, governments, and corporations?

The end result of this investigation will not be a forecast, but rather a collection of plausible scenarios, each one grounded in economic theory and based on first principles.  Although thirty years is a long way to peer into the future, the long time horizon may actually simplify the task in certain ways.  Technological frictions are likely to disappear over time, and individuals are likely to gain an understanding of how privacy markets work, making decisions more in line with their incentives.  Moveover, we can identify at least two modern trends that are likely to persist over the next three decades, motivating economic analysis.

\begin{enumerate}
\item \textit{Algorithms for profiling individuals based on limited data will continue to improve}.  With advances in machine learning and other technologies, more and more aspects of individual behaviors are being mined to extract economically-relevant information.  Future firms and governments will do more with the data they have, leveraging all available information to build a complete picture of our preferences.  The structure of the personal data market depends on how information from different sources combines to generate value.  If a single firm has enough information to build an accurate prediction model, the personal data market may remain fragmented and competitive.  Meanwhile, if many sources of information are needed to predict behavior, the market will tend towards a natural monopoly, which may entail further risks for consumers.

\item \textit{Countermeasures for anonymizing individuals will advance to represent their economic incentives.}  As firms find new sources of information about consumers, technologies will be developed to limit their effects.  Consumers may employ anonymous browsing, search agents, or anonymous payment mechanisms to make purchases while revealing minimal information.  An arms race therefore exists between consumers and the parties that want to track them.  The main limitation on consumer anonymity, however, is likely to be economic rather than technological.  For every consumer that wants to hide their preferences, another one will want to reveal theirs in order to take advantage of a good price offer.  Individuals that employ anonymization will implicitly signal their own characteristics, and  adverse selection may lead to more and more information sharing.

\end{enumerate}

While modeling these trends to identify economic outcomes is an ambitious project, I believe that my research team is in a better position than any other to approach the task.  Our previous work in modeling personal information is already future-oriented, looking beyond individual decisions to identify persistent effects on citizens as a whole.  We have experience working with the a mix of variables that can be usefully abstracted to thirty-year outcomes.  In crafting future scenarios, we will ensure that each one is grounded in first principles and individual incentives.  To complement our economic expertise, we will seek out collaborations with privacy advocates, policy experts, and others who may draw a more complete picture of future outcomes.


\subsection*{Research Methods}

To pursue these research questions, we will employ economic modeling, especially using techniques from game theory.  This methodology is aligned with the strength of my research team, which has successfully applied game theoretic techniques to questions of personal data collection in the past \cite{laskowskigovernment,johnsoncaviar,johnsonprivacy}.  We will complement this work with randomized human-subject experiments, to be conducted online.  By combining these techniques, we hope to gain a more complete and nuanced picture of current trends in personal data collection and how they may affect the state of citizens as a whole. 



\subsubsection*{Game-Theoretic Modeling}
The field of game theory is especially well suited to an exploration of power dynamics because it allows us to bridge the gap from individual incentives to the standing of citizens as a whole.  The structure of a game allows us to identify when an individual's privacy decision imposes negative externalities on other individuals, and what the cumulative impact of many decisions might be.  Once a model has been constructed to describe a scenario involving exchanges of personal data, we derive the model's equilibrium states, which correspond to predictions of behavior. For example, we may consider a game between a set of consumers and a set of corporations in which the corporations sell consumers' personal data to each other.  By characterizing the equilibrium of this game, we can investigate how well consumers fare in response to different governmental policies or corporate incentives.

% That is, we determine the possible outcomes of the model under the assumption that each actor  makes their preferred choice simultaneously or in the specified sequence.  

%A game theoretic model can be applied to scenarios in which the incentives of various actors (e.g. consumers, governments, and/or corporations) are well-approximated by real-valued utility functions, and the consequences of any set of actions by these actors is well-defined (either probabilistically or concretely). When the nature of actors is concretely described (e.g. rational, bounded rational, myopic, etc\dots) these utility functions may often be derived from first principles. For example, we might assume that a government entity acts to maximize its probability of retaining power.  The analysis of a model having this assumption will then tell us something about how a government that has this motivation might behave in different scenarios.  Even though the motivations of real governmental officials are more complex than this -- for example, they may try to behave ethically even if it lowers their odds of staying in power -- such a model can still be useful, giving us an idea of how the balance of power between a government and its citizens might respond to changing surveillance technology.  In short, a model's predictive power depends on the extent to which its actors' assumed nature corresponds with the nature of their present and/or future real-world counterparts.




\subsubsection*{Human-Subject Experiments}

Human-subject experiments have been previously applied to the study of privacy \cite{acquisti2007can,Grossklags:WEIS07,acquisti2004privacy}.  A principle advantage of this technique is the ability to study individual decisions, even when they are not rational in a game-theoretic sense.  Experiments allow us to validate relationships we find in our models, and to supplement our understanding with measures of effect size.


We will use experiments to measure changes in behavior 

Where possible, we will measure changes in behavior, such as the signing of a petition, in addition to attitudes.

Our methodology acknowledges the fact that privacy attitudes and behavior change over time, and it examines factors that may lead to change.


%While the majority of behavioral studies examines privacy behavior in a static sense, our team envisions the use of experiments to directly examine the evolution of privacy attitudes.  
%A major concern in online experimentation is ecological validity.  Subjects may find that the conditions in the online experiment are a poor approximation for real-world phenomena.  We will attempt to mitigate these concerns by basing our experiments on the emerging technology of face recognition.  Although this technology is widely understood, abuses have not been widely covered by news media, giving us an opportunity to shape the information given to our subjects.  We will also focus on relative magnitudes, comparing the effects of different type of information against each other.


%\subsubsection*{Other Data}
%
%Dependent variables  -- occurrences of abuse (indicators for an imbalance of power).
%
%Independent variables -- personal data privacy protections.
%
%
%Look for evidence of abuse 
%
%experiment idea: for lab: vary new story based on factors, ask questions about respondents perceptions on need for privacy protections.
%
%another idea: ask lab participants to read a story ostensibly about security but really measure attitudes about race or religion.
%
%the stories should reveal abuses of power.  want to capture polarized personality types
%
%conduct a series of human subject experiments to capture  how abuses of power influence people's perceptions of privacy practice.


%Various forms of data on the distribution of Governments: e.g. corruptions perception index from transparency international.
%Corporations: better business bureau in the US, various online review systems.
%Individual consumers:  
%Can we use (any) data to put dollar amounts on some of these costs?
%
%variation in strength of technology that seems exogenous, e.g. if us allows microsoft to sell stronger encryption to some country???
%look for events that happen that provide some randomness w.r.t .various variables.
%look at value for different types of data ..
%black market data on credit card numbers etc.. (probly not) 

%\subsection*{Related Work}
%Many papers from Caviar paper.
%8, 11, behavioral studies and 6, 
%9, 10 first sentence
%7 somewhere maybe first sentence, survey paper
%17, maybe first sentence
%15 price discrimination

\subsection*{Broader Impact}

My team believes that personally identifiable data will prove to be one of the most pivotal cybersecurity issues of the next decade.  Like the Center for Long-Term Cybersecurity as a whole, we are motivated by the hope of making a positive impact on cybersecurity practice; we were drawn to the area of personal data after identifying links between privacy and abuses of power against ordinary citizens.  We would be thrilled to join the CLTC community and expect it to enrich our work with diverse perspectives and a sense of common purpose.

Beyond the CLTC community, we hope our work will be interesting to several audiences.  One is the growing community of researchers that apply economic tools to study privacy.  In our exploratory work, we have already advanced the modeling toolkit that that can be applied to describe personally identifiable data, extending two-firm privacy models to account for a continuous distribution of consumers.  We expect our work to yield more methodological advances, beginning with tools to describe repeated privacy decisions over an infinite sequence of time intervals.  In the area of governmental power, we  aim to focus our efforts on parameters like government abuse and dissident speech, bringing economic models more in line with real-world features that activists and ordinary citizens ultimately care about.

Policy makers represent a further audience for our work.  Our pricing models will incorporate regulation as a parameter, allowing us to study the effects of limiting the use or sharing of personally identifiable data.  Government officials may be further interested in how tools for analyzing personal data that are exported overseas may cause other regimes to be more or less corrupt.  In these areas, we hope to create actionable principles that policy makers can use to enhance the position of ordinary citizens against corporations and abusive states.

Finally, we expect our work to generate insights for consumer advocates, human rights advocates, and anyone interested in the standing of individuals in relation to governments and corporations.  A basic finding of our work is that the collection of personal data has consequences for how much surplus individuals give up to corporations, and the risks they face from government abuse.  We therefore hope to give new structure to the debate over privacy, connecting it to other human rights and quantifying the cost to citizens when it is compromised.

%\renewcommand{\bibfont}{\small}
%\begingroup
%\setstretch{0.8}
%\setlength\bibitemsep{0pt}
%%\bibliographystyle{plain}
%%\bibliography{references}
%\printbibliography
%\endgroup
%
\begin{spacing}{0.5}
\setstretch{0.7}
\bibliographystyle{plain}
\bibliography{references}
\end{spacing}



%\newpage

\subsection*{Key Personnel}

Paul Laskowski - Principle investigator, responsible for economic analysis, as well as project scope and direction.

\noindent Benjamin Johnson - Research scientist, responsible for economic analysis, especially mathematical foundations and correctness.

\subsection*{Budget}

\begin{itemize}
\item \$85K/yr: Salary plus benefits for one research scientist
\item \$10K/yr: Travel budget for attending research conferences to present papers 
\item \$5K/yr: Experiment budget primarily for conducting behavioral experiments
\end{itemize}

\noindent This project currently has no other potential funding sources.

\subsection*{Key Project Milestones}

\noindent \textbf{Jan 2016}: Began analysis on first two research questions in parallel\\
\noindent \textbf{April 2016}: First model-based conference paper was submitted\\
\noindent \textbf{July 2016}: Submit second model-based conference paper\\
\noindent \textbf{Sep 2016}: Complete first pilot experiment with human subjects\\
\noindent \textbf{Nov 2016}: Complete first full-scale experiment with human subjects\\
\noindent \textbf{Dec 2016}: Submit first conference paper describing experimental results\\
\noindent \textbf{Feb 2017}: Complete second pilot experiment with human subjects\\
\noindent \textbf{Mar 2017}: Complete second full-scale experiment with human subjects\\
\noindent \textbf{May 2017}: Submit second conference paper describing experimental results\\
\noindent \textbf{Aug 2017}: Submit third model-based conference paper\\
\noindent \textbf{Nov 2017}: Submit experiment-based journal publication.\\
\noindent \textbf{Jan 2018}: Submit fourth model-based conference paper\\
\noindent \textbf{Feb 2018}: Begin analysis on 30-year outlook\\
\noindent \textbf{Mar 2018}: Submit model-based journal publication\\
\noindent \textbf{Apr 2018}: Complete report on 30-year outlook\\



\newpage
\pagenumbering{gobble}
\subsection*{Bios of Key Personnel}

\subsubsection*{Paul Laskowski}

Paul Laskowski is an adjunct assistant professor at the UC Berkeley School of Information.  His research spans the areas of computer networking and microeconomics, with a focus on how policy and network architecture combine to influence consumer choice and technological progress.  At the School of Information, Paul works as part of a team to develop the new Master of Information and Data Science, an online degree program geared towards working data professionals.  In this role, Paul has taught several courses on statistics and programming, and regularly contributes to discussions of curriculum development.  Paul received his Ph.D. from Berkeley's School of Information, and earned his bachelor's in applied mathematics from Harvard University.  Before returning to Berkeley, he was a postdoctoral fellow at the Carnegie Mellon-Portugal program.  Paul has also worked as a developer for Project Indigo, an agent-based modeling environment, and once served, along with Benjamin Johnson, on the founding board of a national gymnastics organization.

\subsubsection*{Benjamin Johnson}

Benjamin Johnson is currently supported by a research scoping award from CLTC. Over the past several years, he has held postdoctoral positions at Carnegie Mellon University, UC Berkeley, Penn State University, and the University of M\"unster. His research publications have spanned multiple areas ranging from quantum complexity theory, to network routing, to user incentives for security investments, to steganography and steganalysis, to timing games, to computer authentication using brainwave signals, to economic aspects of personal data privacy.  %, including twelve different mathematics courses. % including twelve different courses including twelve different courses in mathematics at the university level, and three additional courses at the high school level. 
Benjamin received his Ph.D. from Berkeley's Group in Logic and the Methodology of Science, a cross-departmental program combining math, computer science, and philosophy.  He also holds a masters degree in Mathematics and a bachelors degree in Mathematics with secondary teaching licensure.  In addition to academic research, Benjamin has gained extensive university-level teaching experience in mathematics.  Outside of academia, Benjamin develops web-based software products that are in use by colleges and sports organizations, and he serves on the Board of Directors for the National Association of Intercollegiate Gymnastics Clubs, an organization he founded as a graduate student.

\end{document}
