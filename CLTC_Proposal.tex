\documentclass{article}

\usepackage{fullpage}
\usepackage{amsmath}
\usepackage{amssymb}
\usepackage{amsthm}
\usepackage{multirow}
\usepackage{graphicx}
\usepackage{mathtools}
\usepackage{pifont}
\usepackage{color}


\newtheorem{theorem}{Theorem}
\newtheorem{proposition}{Proposition}
\newtheorem{corollary}{Corollary}
\newtheorem{conjecture}{Conjecture}
\newtheorem{lemma}{Lemma}
\theoremstyle{definition}
\newtheorem{definition}{Definition}

\DeclareMathOperator*{\argmin}{argmin}

%%%% New commands %%%%
\def\Pr{\mbox{Pr}}
\def\PR{\mbox{Pr}}
\newcommand{\yes}{\operatorname{yes}}
\newcommand{\no}{\operatorname{no}}
\newcommand{\tilf}{\tilde{f}}
\newcommand{\bias}{\tilf}
\newcommand{\B}{\mbox{Big}}
\newcommand{\vect}[1]{\boldsymbol{#1}}

\newcommand{\tick}[0]{\checkmark}

\newcommand{\nicolasc}{\ding{110}\ding{43}\textcolor{blue}}
\newcommand{\benjamin}{\ding{110}\ding{43}\textcolor{green}}

\title{(Im)balances of Power in the Age of Personal Data}

\begin{document}

\maketitle

Principal Investigator: Paul Laskowski

What truly matters when protecting our personal data?   We live in a world where our purchases, browsing history, and even mouse clicks are logged by computer systems and increasingly analyzed.  While businesses and governments have identified tremendous opportunities in data like this, how can we measure the corresponding costs associated with the privacy we are losing?  

One avenue is to consider individual decisions.  Economic models let us predict how much firms can price discriminate, based on their knowledge of a consumer.  Surveys give us insight into how much consumers value different types of personal information.  Meanwhile, behavioral studies reveal how consumers may give up their data for simple convenience, despite how much they say they value keeping it secret.

Yet a focus on individual decisions misses a large part of what makes privacy valuable, if not the most important part.  Writings by privacy commentators reveal a deep concern over how mass collection of personal data will affect consumers as a whole, and the balance of power between consumers, governments, and corporations.  The Electronic Frontier Foundation's main page on privacy argues that ``national governments must put legal checks in place to prevent abuse of state powers."  Similarly, the Bloomberg ``Wired for Repression'' series argues, ``In authoritarian countries throughout the Middle East and North Africa, Western surveillance tools have empowered repression.''  As Valentino-Devries, Singer-Vine and Soltani recently argued in the Wall Street Journal, increasingly targeted pricing by companies ``diminishes the Internet's role as an equalizer.'' 

Though sweeping in scope, we must grapple with risks like these if we are to assess the true cost of collecting personal data.  It is not enough to look at individual decisions in isolation because the cumulative impact of individual decisions can change technology and institutions.  Profiling individual consumers may become easier as more data is available.  Enhanced data collection may embolden repressive elements in government, increasing the risk for dissidents.  Consumers may face reduced privacy choices if companies can neglect privacy conscious individuals.

This proposal therefore examines the role of citizens as a whole, especially in relation to governments, corporations, and other actors.  We seek to measure how individual privacy choices may tilt the playing field that citizens face.   The ultimate goal is a more complete and more accurate assessment of the costs of personal data collection, one that may benefit the public dialogue about data collection and inform future policy discussions.

Research Question 1: How does the collection of personal data affect the balance of power between citizens and the state?

Modern tools for data collection give considerable advantage to governments in monitoring their own citizens and foreign nationals.  Once regarded chiefly as a problem of third-world autocracies, many western governments, including the United States, now face widespread scrutiny over the potential abuse of surveillance technology. � Document leaks and news reports have revealed an extensive surveillance apparatus, fueling a lively debate about the proper role of intelligence agencies in society. Proponents maintain that surveillance programs are instrumental in preventing terrorist attacks. Opponents counter that spying erodes individual privacy and facilitates totalitarian states.

In an exploratory study, I worked with colleagues at UC Berkeley to examine this issue through the lens of government incentives.  While abusive governments will certainly use surveillance technology to their advantage, a more subtle question is whether technology leads governments to be more or less repressive in the first place.  Using a stylized game theoretic model, we examined how surveillance technology influences the incentives for a government to abuse its power.

To support our results, my team was careful to separate the use of surveillance technology from its abuse.  This reflects our understanding that the collecting of citizen data by governments is not inherently corrupt.  Moreover, many applications of surveillance, including for law enforcement and security, are uncontroversial.  We therefore adopted a working definition of abuse, as the use of power in excess of moral or ethical standards.  

Our game theoretic model suggests that surveillance technology incentivizes greater abuse of power by governments.  Under surprisingly general conditions, stronger surveillance technology will lead a rational government that wishes to stay in power to abuse its power more.   Although we recognize that well-meaning government officials may ignore such incentives in order to behave ethically, we believe that incentives matter for explaining differences in abuse levels. 

While these initial results are promising, they open myriad directions for further study.  How does a rational government behave in the presence of multiple threats to its power, say a radicalized insurgency and a political opposition?  What explains differing levels of abuse by different governments?  What types of counter-technologies can tilt the playing field in favor of more benign governance?  By answering these questions, this initiative will shed light on how surveillance relates to government abuse and what technologies are most effective in promoting ethical governance. 

Research Question 2: How does the collection of personal data affect the balance of power between consumers and corporations?

A second line of inquiry concerns the balance of power between consumers and corporations.  While individual privacy decisions by consumers have been thoroughly examined, we understand far less about how the privacy options that consumers face are generated, and how they may change in response to more advanced tracking technology.

One area of rapid change is in the sharing of consumer data between firms.  Recognizing the potential opportunity that can come from linking consumer decisions across different merchants, a new generation of firms is appearing to mediate the sharing of consumer data.  In an early paper, I examined a simple scenario in which consumers purchased goods from two different firms in sequence.  When firms are allowed to sell their purchase data to each other, they can leverage the information they learn about consumer to set discriminatory prices.

Like previous studies in this lineage, we find that the ultimate consequences for consumers depend on whether they are myopic, regarding each purchase in isolation, or fully rational, choosing products strategically in light on how purchases may affect future price offers.  While myopic consumers always lose out when firms share information, sophisticated consumers will mitigate this effect by limiting the information they reveal through purchases.

Methods

Related Work

Broader Impact

Budget


\end{document}
